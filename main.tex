\documentclass[12pt]{article}
\usepackage{graphicx}
\usepackage{amsfonts}
\usepackage{fancyhdr}
\usepackage{comment}
\usepackage{amsmath}
\usepackage[a4paper, top=2.5cm, bottom=2.5cm, left=2.2cm, right=2.2cm]%
{geometry}
\usepackage{times}
\usepackage{amsmath}
\usepackage{changepage}
\usepackage{amssymb}
\usepackage{fixltx2e}
\usepackage{enumerate}
\usepackage{graphicx}
\usepackage{float}
\newtheorem{theorem}{Theorem}
\newtheorem{acknowledgement}[theorem]{Acknowledgement}
\newtheorem{algorithm}[theorem]{Algorithm}
\newtheorem{axiom}{Axiom}
\newtheorem{case}[theorem]{Case}
\newtheorem{claim}[theorem]{Claim}
\newtheorem{conclusion}[theorem]{Conclusion}
\newtheorem{condition}[theorem]{Condition}
\newtheorem{conjecture}[theorem]{Conjecture}
\newtheorem{corollary}[theorem]{Corollary}
\newtheorem{criterion}[theorem]{Criterion}
\newtheorem{definition}[theorem]{Definition}
\newtheorem{example}[theorem]{Example}
\newtheorem{exercise}[theorem]{Exercise}
\newtheorem{lemma}[theorem]{Lemma}
\newtheorem{notation}[theorem]{Notation}
\newtheorem{problem}[theorem]{Problem}
\newtheorem{proposition}[theorem]{Proposition}
\newtheorem{remark}[theorem]{Remark}
\newtheorem{solution}[theorem]{Solution}
\newtheorem{summary}[theorem]{Summary}
\newenvironment{proof}[1][Proof]{\textbf{#1.} }{\ \rule{0.5em}{0.5em}}

\newcommand{\Q}{\mathbb{Q}}
\newcommand{\R}{\mathbb{R}}
\newcommand{\C}{\mathbb{C}}
\newcommand{\Z}{\mathbb{Z}}

\begin{document}

\title{SIE 545: Fundamentals of Optimization \\Homework 5}
\author{Martin Manuel Lopez \\lopez9@email.arizona.edu \\Systems and Industrial Engineering}
\date{December 4, 2018}
\maketitle
\section{BSS 6.3}
    $\phi(x,y)$ is a continuous function $x \in X \subseteq R^n $ and $y \in Y \subseteq R^m$. We need to provide the following proof:\\
        \begin{align*}
            &\sup_{y \in Y} \inf_{x \in X} \phi(x,y) \leq \inf_{x \in X} \sup_{y \in Y} \phi(x,Y)\\
        \end{align*}
    We use the following definition for $\phi$: \\
        \begin{align*}
            &\inf_{x \in X} \phi(x,y) \leq \phi(x,y) \leq \sup_{y \in Y} \phi(x,y)\\
        \end{align*}
    Let us take the left hand side of the equation and define it as:\\
        \begin{align*}
            &g(y) = \inf_{x \in X} \phi (x,y) \\ 
        \end{align*}
    We then take the right hand side of the same equation and define as:\\ 
        \begin{align*}
            &h(x) = \sup_{y \in Y} \phi (x,y)
        \end{align*}
    We reformulate the equation: \\
        \begin{align*}
            &\sup_{y \in Y} g(y) \leq \inf_{x \in X} h(x)\\
        \end{align*}
    We finaly replace $g(y)$ and $h(x)$\\
        \begin{align*}
             &\sup_{y \in Y} \inf_{x \in X} \phi(x,y) \leq \inf_{x \in X} \sup_{y \in Y} \phi(x,Y)\\
        \end{align*}
\section{BSS 6.10}
    Let us consider the following problem: \\
        \begin{align*}
            &\min x_1^2 + x_2^2\\
            &s.t.\\
            &x_1 + x_2 - 4 \geq 0\\
            &x_1 \geq 0\\
            &x_2 \geq 0 \\
        \end{align*}
    We will reformulate the problem:\\
        \begin{align*}
            &\min x_1^2 + x_2^2\\
            &s.t.\\
            &4-x_1-x_2 \leq 0\\
            &x_1 \geq 0\\
            &x_2 \geq 0 \\
        \end{align*}
    a.) We must verify that the optimal solution is $\bar x = (2,2)^T$ with $f(\bar x) = 8$. \\ 
        \begin{align*}
            &\phi (x,u) = x_1^2 + x_2^2 + u (4 - x_1 - x_2)\\
            &L(x) = \phi (x,y) = (x_1 - \frac{u}{2})^2 + u (x_2 - \frac{u}{2})^2 - \frac{u^2}{2} + 4u\\
        \end{align*}
    We utilize KKT necessary and sufficient conditions: \\
        \begin{align*}
            &\nabla L(\bar x) = \nabla f(\bar x) + \sum_{i \in I} \bar u_i g_i (\bar x) = 0\\ 
            &\nabla f(\bar x) 
                \begin{bmatrix}
                    2x_1\\
                    2x_2\\
                \end{bmatrix} = 
                \begin{bmatrix}
                    4\\
                    4\\
                \end{bmatrix}
        \end{align*}
    We then have the following : \\ 
        \begin{align*}
            &L(\bar x) = \begin{bmatrix}
                4\\
                4\\
            \end{bmatrix} + 
            u_1 
            \begin{bmatrix}
                -1\\
                -1\\
            \end{bmatrix} + 
            u_2 
            \begin{bmatrix}
                1\\
                0\\
            \end{bmatrix} + 
            u_3 
            \begin{bmatrix} 
                0\\
                1\\
            \end{bmatrix} = 
            \begin{bmatrix}
                0\\
                0\\
            \end{bmatrix}
        \end{align*}
    We obtain the following CS: \\ 
        \begin{align*}
            &4 - u_1 + u_2 = 0 \\
            &4 - u_1 + u_3 = 0 \\ 
            &\Rightarrow\\
            &u_1 = 4 + u_2\\
            &4 - 4 - u_2 + u_3 = 0\\
            &\Rightarrow\\
            &u_2 = u_3\\
        \end{align*}
    We let $u_3 = u_2 = 0$ thus we get $u_1 = 4$ and thus $u_{i \in I} \geq 0$\\ 
    We check the Hessian Matrix to make sure that it meets the KKT sufficient conditions:\\
        \begin{align*}
            &\nabla^2 L(\bar x) = 
                \begin{bmatrix}
                    2 & 0 \\
                    0 & 2 \\
                \end{bmatrix}
            &det|\nabla^2 L(\bar x) | = 4 > 0\\
        \end{align*}
    We thus have the Hessian Matrix as Positive Definite and meet the KKT necessary and sufficient conditions based on the first and second order theorems. We thus conclude that $\bar x = (2,2)^T $ and get the optimal function value of $f(\bar x) = 8$. \\
    b.) We get the Langrange function: 
    \begin{align*}
        & \theta (u) = \inf_{x \in X} \phi (x,u) = \inf_{x \in X} [(x_1 - \frac{u}{2})^2 + (x_2 - \frac{u}{2})^2 - \frac{u^2}{2} + 4u]\\
        &\Rightarrow\\
        &\theta (u) = \min_{x \in X} [(x_1 - \frac{u}{2})^2 + (x_2 - \frac{u}{2})^2 - \frac{u^2}{2} + 4u]\\
        &\Rightarrow\\
    \end{align*}
    We have the following condition in finding the optimal solution to the minimizing problem:\\
        \begin{align*}
            &u_i \geq 0\\
        \end{align*}
    We now see that the follwing Dual Problem is : \\
        \begin{align*}
            & \max \theta (u) = \max - \frac{u^2}{2} + 4u\\
            &u \geq 0\\
        \end{align*}
    We now solve for Dual Problem to confirm that max value is equal to Primal and there is no duality gap when we solve and find $u = 4$: \\ 
        \begin{align*}
            &\theta(4) = -\frac{16}{2} + 16 = 16 - 8 = 8 \\
        \end{align*}
    c.) We must solve the Dual problem using the cutting plane algorithm:\\
    
\end{document}